%!TEX root = ../report.tex

\begin{document}
    \chapter{Conclusions}

    \section{Contributions}
    - The solution presented in this work can be used and improved for the registration of a 3D Model with a point cloud using different kinds of features 

    \section{Lessons learned}
    - How to make a research in a specific field. \par
    - How to work in a project that already has many specifications defined. \par
    - How to work with third-party libraries in order to ease the implementation of a project. \par
    - How to adapt yourself to the changes in the requirements of a project. \par
    

    \section{Future work}
    - Improve the angle detection.
    - The direction of the corners can be added to improve the performance.
    - More features can be added.
    - There are two possible options to improve the performance of the system (reduce the time of registration) \par
    - One can work with 3D features and find directly a 3D transformation that aligns the CityGML-Model and the point cloud. \par
    - a Branch-and-Bounding \cite{Bazin_2013_abranchandbound,Breuel_2003_implementation,Brown_2015_globally,Brown_2019_afamily} approach could be applied 
    to find a transformation that aligns the projections of the CityGML model and the point cloud onto the XY-plane or to find such a transformation directly
    with 3D features such as lines or corners. \par
\end{document}
