%!TEX root = ../report.tex

\begin{document}
    \chapter{Conclusions}

    The results obtained from the proposed solution are successful. 
    Although the proposed method provides a coarse registration, it is significant for the A-DRZ project’s proposes. 
    The registration method is ready to be included in the A-DRZ software of the Fraunhofer IAIS, 
    and whenever there are new point clouds, the method could be further tested and improved.

%-------------------------------------------------------------------------------
%	Contributions
%-------------------------------------------------------------------------------
    \section{Contributions}
        This work’s main contribution is a coarse registration method for a 3D model and a point cloud 
        that converts the 3D registration task into a 2D registration task and works based on the intersection of lines and their angle information. 
        Furthermore, the solution presented can be adapted to be used with different kinds of 2D features.
    
        In addition to this work’s scientific contributions, 
        it also contributes to the A-DRZ project as an automatic registration method for their visualization system, 
        helping rescue services to obtain an overview of the current status of possible dangerous situations.
    

%-------------------------------------------------------------------------------
%	Lessons learned
%-------------------------------------------------------------------------------
    % \section{Lessons learned}
    %     When doing research, it is essential to keep the eyes open and try to link ideas between different papers. 
    %     That could help to combine existing methods to come up with improved solutions for a problem. 
    %     Moreover, the research should not necessarily be focused on one specific field, 
    %     and sometimes it is needed to take a look at ideas of different fields.

    %     The use of appropriate libraries can simplify and reduce the work implementing long solutions, like the one presented in this work. 
    %     Moreover, they might also reduce the running time of the system because their implementation is usually optimized. 
    
    %     When working within the limits of an already existing project, such as the A-DRZ, 
    %     one should search for a solution that fits the project’s conditions. 
    %     Furthermore, one should be ready to adapt the work to sudden changes.
    
%-------------------------------------------------------------------------------
%	Future work
%-------------------------------------------------------------------------------
    \section{Future work}
        The detection of lines and their intersection is the most crucial part of the solution proposed 
        since these are the features used for the finding of correspondences conducted by the Mixed Integer Linear Program. 
        Therefore, improving this task’s performance will benefit the registration’s global performance, although it is not easy.

        Another improvement that could be made is to increase the information in the angles detected. 
        Not just the position of the angle can be used in the Mixed Integer Linear Program, but also the orientation of the angle, 
        i.e., the orientation of the sum of the line segments that form the angle.
        Moreover, other kinds of features can be included. For example, the lines or even other corners that are not necessarily intersections of lines.

        Two additional options could be tried to improve the system’s performance in the future. 
        Firstly, one could work with 3D features and find directly a 3D transformation that aligns the CityGML model and the point cloud. 
        Secondly, a Branch-and-Bounding approach \cite{Bazin_2013_abranchandbound,Breuel_2003_implementation,Brown_2015_globally,Brown_2019_afamily} 
        could be applied to search for a transformation that aligns the projections of the CityGML model and the point cloud onto the xy-plane 
        or to search such a transformation directly with 3D features such as planes, lines, or corners.  

\end{document}
