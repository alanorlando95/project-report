%!TEX root = ../report.tex

\begin{document}
    \chapter{Conclusions}

    The results obtained from the proposed solution are successful. 
    Although they just provide a coarse registration, this is significant for the proposes of the A-DRZ project.
    The registration method is ready to be included in the A-DRZ software of the Fraunhofer IAIS,
    and whenever there are new point clouds, the method could be further tested and imporved.

%-------------------------------------------------------------------------------
%	Contributions
%-------------------------------------------------------------------------------
    \section{Contributions}
        The main contribution of this work is a coarse registration method for a 3D model and a point cloud
        that converts the 3D registration task into a 2D registration task and works based on the intersection of lines and their angle information.
        Furthermore, the solution presented can be adapted to be used with different kind of 2D features. 

%-------------------------------------------------------------------------------
%	Lessons learned
%-------------------------------------------------------------------------------
    \section{Lessons learned}
        - Opend3D offers a user-friendly interface to work with point clouds and 3D models. \par
        -

        - How to make a research in a specific field. \par
        - How to work in a project that already has many specifications defined. \par
        - How to work with third-party libraries in order to ease the implementation of a project. \par
        - How to adapt yourself to the changes in the requirements of a project. \par
    
%-------------------------------------------------------------------------------
%	Future work
%-------------------------------------------------------------------------------
    \section{Future work}
        The detection of lines and their intersection is the most crucial part of the solution proposed since these are the features used 
        for the finding of correspondences conducted by the Mixed Integer Linear Program. 
        Therefore, an improvement in the performance of this task will benefit the global performance of the registration.

        Another improvement that could be made is to increase the information in the angles detected. 
        Not just the position of the angle can be used in the Mixed Integer Linear Program, but also the orientation of the angle, 
        i.e., the orientation of the sum of the line segments that form the angle.
        Moreover, other kinds of features can be included. For example, the lines or even other corners that are not necessarily intersections of lines.

        Two additional options could be tested to improve the system’s performance in the future. 
        Firstly, one could work with 3D features and find directly a 3D transformation that aligns the CityGML model and the point cloud. 
        Secondly, a Branch-and-Bounding approach \cite{Bazin_2013_abranchandbound,Breuel_2003_implementation,Brown_2015_globally,Brown_2019_afamily} 
        could be applied to search for a transformation that aligns the projections of the CityGML model and the point cloud onto the xy-plane 
        or to search such a transformation directly with 3D features such as planes, lines, or corners.  

\end{document}
