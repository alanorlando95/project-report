%!TEX root = ../report.tex

\begin{document}
    \chapter{State of the Art}
    \label{chap:State of the Art}
    Point cloud registration algorithms are normally classified into two major categories: global (coarse) registration and local (fine) registration \cite{Quan_2020_com,Kim_2011_fully}.
    Coarse registration algorithms attempt to find a transformation that approximately maps the source point cloud to the target point cloud. 
    Fine registration algorithms attempt to refine an initial transformation to map the source point cloud to the target point cloud.
    There are some approaches that combine both of these methods, they first apply a global method and then refine the results with the help of a local method. 
    We can classify these approaches into a 3rd category of hybrid methods. 
    
    \par
    
    For the purpose of this work we can classify point cloud registration algorithms into two major categories:
    point-to-point and point-to-model.
    Point-to-point registration algorithms attempt to align two point clouds, while point-to-model registration algorithms attempt to align a point cloud with a 3D model.

    \section{Point-to-point registration}

        \subsection{Global methods}

        RANSAC \cite{Fischler_1981_RANSAC} can also be used to perform the registration task but is time-consuming due to the randomness
        of the RANSAC algorithm. Therefore, Pankaj et al. \cite{Pankaj_2015_arobust} present a guided sampling improvement to RANSAC,
        to reduce the registration time, sorting the correspondences obtained from a previous 3D feature matching according to its quality.
        
        Following the same line, S. Quan and J. Yang \cite{Quan_2020_com} propose a method that helps the RANSAC algorithm 
        sampling the best point correspondences. Their approach first detects keypoints using Harris3D, then it matches the keypoints
        by using the Local Voxelized Structure (LoVS) descriptor. After that, the set of correspondences is reduced using the 
        Nearest Neighbor Similarity Ration (NNSR) and the compatibility score of the remaining correspondences with the rigidity and 
        DSP constraint. Finally, the correspondences with the best compatibility scores are using to generate transformations and 
        the best one (the one with the maximum number of inliers) is the final result. Unfortunately, the RANSAC-based approaches are sensitive to outliers.

        Another less common approach for coarse registration is proposed by Sakakubara et al. \cite{Sakakubara_2007_automatic}.
        This coarse registration approach uses Mixed Integer Linear Programming to find pairs of corresponding points that are then 
        used to find a coarse transformation that aligns two different point clouds.
        This approach finds a global optimal result without using the values of the invariant features, 
        it adjusts the error tolerance depending on the accuracy of the given data and 
        it finds the best correspondences between the points.
        Nevertheless, Mixed Integer Linear Programming problems are NP-hard. Therefore, this approach is excessively time-consuming.

        \subsection{Local methods}

        Besl and N. D. McKay \cite{Besl_1992_amethod} propose the Iterative Closest Point (ICP) algorithm to perform registration 
        between point sets, line segments, implicit curves, parametric curves, triangles, implicit surfaces, and parametric surfaces.
        ICP iteratively finds and updates correspondences between the desired sets based on an initial approximation to the solution and 
        the spatial distance.
        ICP is still the standard algorithm for point cloud registration. However, to perform well it requires a good initial
        approximation of the solution.

        \subsection{Hybrid methods}

        The approach proposed by Jun Lu et al. \cite{Lu_2019_4pcsicp} obtains a consistent four-point set by using Super4PCS \cite{Mellado_2014_super4pcs}, 
        then it creates a neighborhood ball with each point as the center and obtains the overlapping regions as the intersections
        of these neighborhood balls. Finally, ICP is applied to the overlapping regions. The main drawback of this method is that 
        there should exist overlapping regions in order to perform the registration task.

        Because of the increasing success of Deep Learning in 2D applications, there have been attempts to use deep neural networks
        to solve the point cloud registration problem.
        DeepICP \cite{Lu_2019_deepicp} is the first end-to-end learning-based point cloud registration framework.
        DeepICP makes use of other networks to build a complete framework.
        The first step of DeepICP is the Deep Feature Extraction (PointNet+ \cite{Qi_2017_pointnet}), which extracts features from the point clouds.
        In the second step, the Pont Weighting (3DFeatNet \cite{Yew_2018_3dfeat}) selects the relevant points.
        In the third step, the selected points of the target point cloud are sampled using the initial transformations.
        In the fourth step, Deep Feature Embedding is used to obtain more specific descriptors of the points.
        In the final step, the corresponding pair of points are generated with a 3D Convolutional Neural Network
        to compute an improvement of the initial transformation.
        Unfortunately, Methods based on deep learning generally suffer from issues regarding the generalization ability and 
        the requirement of massive training data \cite{Quan_2020_com}.

        Another approaches does not use a neural network to solve the complete point cloud registration problem but a part of it.
        LORAX is an algorithm proposed in \cite{Elbaz_2017_3dpoint} that selects some sets of points called super-points and uses a low-dimensional descriptor 
        generated by an autoencoder to describe the structure of each super-point.
        The sets of superpoints of both pointclouds are then matched using the euclidean distance between the descriptors.
        Finally, a coarse registration is computed using a RANSAC procedure, followed by a fine registration using ICP.
        The main contribution of LORAX is the use of an autoencoder to extract features of 3D point clouds. 
        Nevertheless, the implementation is not optimized for real-time performance. Nevertheless, this approach is not 
        proved to be fast for big point clouds.

    \section{Point-to-model registration}

        \subsection{Sampling methods}
        In order to register a 3D model with a point cloud, the simpliest idea is to sample points from the 3D model to generate a second point cloud
        and then perform one of the point-to-point registration methods. 

        Kim et al. \cite{Kim_2011_fully} register a 3D CAD model of a construction site with a point cloud by sampling points from the 3D model 
        and then applying a point-to-point registration algorithm. The registration consists of three main steps. First, 
        the point clouds are re-sampled to the same resolution. Second, PCA is used to identify the principal axes of the data 
        to obtain a coarse registration based on these axes. Third, fine registration is performed with the Levenberg-Marquardt
        iterative closest point (LM-ICP) algorithm \cite{Fitzgibbon_2003_robust}. 
        
        Kim et al. \cite{Kim_2013_fully} is an extension of their previous work \cite{Kim_2011_fully}
        In this approach, they just add a noise filter step to the pre-processing of the point cloud in order to improve the registration results.
        The rest of the steps remain the same.
        However, using PCA for coarse registration will not assure a good performance when the form of the building is close to a cube.
        
        \subsection{Local methods}

        Other approaches work directly with the information provided in the 3D models to perform the registration task.

        Li and Song \cite{Li_2015_amodified} propose two extensions to the most popular local registration method, the Iterative Closest Point (ICP) algorithm, 
        to be able to use it with an STL-file (3D model) and a point cloud.
        The first of these extensions is called ICP-STL, and its main modification consists of finding the corresponding triangle mesh of the STL file of the points 
        in the point cloud, and then the closest point in the triangle mesh to each point of the point cloud. The rest of the ICP algorithm remains the same.
        The second modification is called ICP-DAF (ICP dynamic adjustment factor), and it aims to reduce the time of the ICP-STL method. 
        ICP-DAF adds a dynamic adjustment factor to adjust the transformation parameters dinamically.
        The disadvantage of this method is the same disadvantage present in the ICP algorithm, to perform well it requires a good initial approximation of the solution.

        In the specific case of CityGML, \cite{Goebbels_2019_icpcitygml} proposes an extension of the well-known Iterative Closest Point algorithm (ICP) 
        to perform point-to-model registration based on the point-to-plane registration.
        This approach only considers points on the perimeter of the buildings.
        The main contribution of this approach is the speed. Point-to-model ICP is faster than the point-to-plane ICP.
        Nevertheless, CityGML models used do not contain terrain information. 
        Therefore, the ground plane is the lowest point of the building and some walls might intersect with the real terrain, increasing the error of the algorithm.
        Another disadvantage of this method is that it expects the point cloud to be coarsely register with the model.

        Another approach by Goebbels et al. \cite{Goebbels_2018_linebased} explores the possibility to find features that help to perform the registration.
        The approach first rotates the point cloud to align the walls to the z-axis. Then it projects all the points onto the 
        xy-plane to obtain a 2D image of the point cloud viewed from above. After that, it detects the corners or lines in the 2D image,
        and having the corners or lines of the CityGML model it applies a Mixed Integer Linear Program to perform the registration.
        Nonetheless, the point cloud has to be already coarsely registered in order to obtain a successful and fast result.

        An extension of the work in \cite{Goebbels_2018_linebased} has been made by Goebbels et al. \cite{Goebbels_2018_alinear}, 
        where the steps to perform the registration are the same.
        However, in the last step, a Linear Program is used to fine-tune the coarse result. 
        Unfortunately, the results of this method depend on the value of an error bound, which affects the generalization of the algorithm.
        Moreover, it has the same disadvantage as the previous approach: 
        the point cloud has to be already coarsely registered in order to obtain a successful and fast result.
    
        \subsection{Hybrid methods}

        
        

    \section{Limitations of previous work}
        - They does not offer a concrete solution for the coarse registration of a model with a point cloud. \par
        - The sampling of a model to obtain a point cloud and then register it with another point cloud, do not leverage the information contained in the model. \par
        - To train a neural network for this is not an option due to the size of the dataset available for our project.

\end{document}
