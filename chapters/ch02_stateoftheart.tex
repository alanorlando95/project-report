%!TEX root = ../report.tex

\begin{document}
    \chapter{State of the Art}
    \label{chap:State of the Art}
    Point cloud registration algorithms are typically classified into two major categories: global (coarse) registration and local (fine) registration \cite{Quan_2020_com,Kim_2011_fully}.
    Coarse registration algorithms attempt to find a transformation that approximately maps the source point cloud to the target point cloud. 
    Fine registration algorithms attempt to refine an initial transformation that maps the source point cloud to the target point cloud.
    Some approaches combine both of these methods; they first apply a global method and then refine the results with the help of a local method. 
    We can classify these approaches into a third category called hybrid methods. 
    
    For the purpose of this work, we classify point cloud registration algorithms into two other major categories:
    point-to-point and point-to-model.
    Point-to-point registration algorithms attempt to align two point clouds, while point-to-model registration algorithms attempt to align a point cloud with a 3D model.

    The approaches that do not use point cloud data or 3D models, but are still related to the registration problem, are classified into the category of others.
    \autoref{fig:classification} shows an overview of the classification of the papers reviewed.
    
    \begin{figure}[H]
    \centering
        \begin{tikzpicture}[->, >=latex]
            \matrix [column sep=7mm, row sep=5mm,
            nodes={rectangle, draw}] { 
            & &
              \node (rm) [fill=green, anchor = base west, xshift=-0.51cm, yshift=-0.5cm] {Registration methods}; \\
              \node (ptp) [fill=orange, anchor=base west] {point-to-point}; & &
              \node (ptm) [fill=orange, anchor=base west] {point-to-model}; & &
              \node (ot) [fill=orange, align=center, right= of ptm.north east, anchor= north west] {others\\ \cite{Breuel_2003_implementation, Bazin_2013_abranchandbound, Brown_2015_globally, Brown_2019_afamily, Windheuser_2011_largescale}}; \\
              \node (ptpg) [fill=cyan, anchor=base west, align=center] {global\\ \cite{Fischler_1981_RANSAC, Pankaj_2015_arobust, Quan_2020_com, Sakakubara_2007_automatic, Wang_2019_deepclosest, Sarode_2019_oneframework, Huang_2020_feature} }; & &
              \node (ptms) [fill=cyan, anchor=base west, align=center] {sampling\\ \cite{Kim_2011_fully, Kim_2013_fully} }; \\
              \node (ptpl) [fill=cyan, anchor=base west, align=center] {local\\ \cite{Besl_1992_amethod, Segal_2009_generalizedicp}}; & &
              \node (ptml) [fill=cyan, anchor=base west, align=center] {local\\ \cite{Li_2015_amodified, Goebbels_2019_icpcitygml, Goebbels_2018_alinear, Goebbels_2018_linebased}}; \\
              \node (ptph) [fill=cyan, anchor=base west, align=center] {hybrid\\ \cite{Yang_2016_goicp, Lu_2019_4pcsicp, Lu_2019_deepicp, Elbaz_2017_3dpoint}}; \\
            };
            \draw[thick] (rm) -| (ptp);
            \draw[thick] (rm) -- (ptm);
            \draw[thick] (rm) -| (ot);
            \draw[thick] (ptp.west) -- ++(-0.5, 0) |- (ptpg.west);
            \draw[thick] (ptp.west) -- ++(-0.5, 0) |- (ptpl.west);
            \draw[thick] (ptp.west) -- ++(-0.5, 0) |- (ptph.west);
            \draw[thick] (ptm.west) -- ++(-0.5, 0) |- (ptms.west);
            \draw[thick] (ptm.west) -- ++(-0.5, 0) |- (ptml.west);
        \end{tikzpicture}
        \caption{Classification of the State of the Art.}
        \label{fig:classification}
    \end{figure}

%-------------------------------------------------------------------------------
%	Point-to-point registration
%-------------------------------------------------------------------------------
    \section{Point-to-point registration}
    %-------------------------------------------------------------------------------
    %	Global methods
    %-------------------------------------------------------------------------------  
        \subsection{Global methods}

        RANSAC \cite{Fischler_1981_RANSAC} is a well-known algorithm that can be used to perform the registration task. 
        Generally speaking, given a set of already computed correspondences, the algorithm consists of 2 main steps that are repeated iteratively. 
        In the first step, a number of correspondences, depending on the degrees of freedom of the required transformation, 
        are randomly selected and used to compute a perfect transformation. 
        In the second step, the number of outliers is counted. 
        If the value is under a given percentage, then the algorithm terminates. Otherwise, it is repeated from the first step. 
        Another stop condition is the number of iterations. 
        If the value is above a given limit, then the algorithm stops, and the transformation with the lower percentage of outliers is selected.
        
        Unfortunately, RANSAC is time-consuming due to its randomness.
        Therefore, Pankaj et al. \cite{Pankaj_2015_arobust} present a guided sampling improvement to RANSAC
        to reduce the registration time, sorting the correspondences obtained from a previous 3D feature matching according to their quality.
        
        Following the same line, S. Quan and J. Yang \cite{Quan_2020_com} propose a method that helps the RANSAC algorithm 
        sample the best point correspondences. Their approach first detects keypoints using Harris3D, and then it matches the keypoints
        by using the Local Voxelized Structure (LoVS) descriptor. 
        After that, the set of correspondences is reduced using the Nearest Neighbor Similarity Ratio (NNSR) and the compatibility score of the remaining correspondences. 
        The compatibility score is a function of two constraints: the rigidity constraint and distance between salient points (DSP) constraint, 
        both of them defined by the authors.
        Finally, the correspondences with the best compatibility scores are used to generate transformations,
        and the best one (the one with the maximum number of inliers) is the final result.

        Another less common approach for coarse registration is proposed by Sakakubara et al. \cite{Sakakubara_2007_automatic}.
        This coarse registration approach uses Mixed Integer Linear Programming to find pairs of corresponding points %that are then 
        used to find a coarse transformation that aligns two different point clouds.
        This approach finds a global optimal result without using the values of invariant features.
        It adjusts the error tolerance depending on the accuracy of the given data and 
        finds the best correspondences between the points.
        Nevertheless, Mixed Integer Linear Programming problems are NP-hard. Therefore, this approach is excessively time-consuming.

        Others have proposed the use of neural networks to solve the global registration problem.
        Wang and Solomon \cite{Wang_2019_deepclosest} propose a learning-based method called Deep Closest Point (DCP).
        The approach proposed first uses PointNet \cite{Qi_2017_pointnetdeep} or DGCNN \cite{Wang_2019_dynamic} to embed the point clouds.
        Then, it uses an attention-based module to encode the output of the first part.
        Finally, it estimates the transformation using a differentiable SVD layer.

        Sarode et al. \cite{Sarode_2019_oneframework} present a registration framework that also uses PointNet to extract features from the point clouds.
        The PointNet features are then aligned using another network. The process is then repeated a given number of times to improve the registration result.
        
        Ding and Feng \cite{Ding_2019_deepmapping} present another deep learning method called DeepMapping,
        whose main structure is constituted of two deep neural networks.
        The first network is a localization network, whose output is the pose estimation for the input point clouds,
        and the second network is a map network that estimates the occupancy status of the input locations to compute a loss function that reveals
        the registration quality.

        More recently, Huang et al. \cite{Huang_2020_feature} propose a learning method that can be trained in a semi-supervised or unsupervised way.
        The proposed framework is composed of an encoder that extracts features from the input point clouds
        and a decoder that interprets the extracted features to compute a projection error that serves as input for an algorithm that estimates
        a transformation increment. 
        This increment is then employed to update the parameters of the transformation and transform one of the point clouds.
        The whole process is repeated until the loss computed is less than some desired value.

    %-------------------------------------------------------------------------------
    %	Local methods
    %-------------------------------------------------------------------------------  
        \subsection{Local methods} 

        Besl and N. D. McKay \cite{Besl_1992_amethod} propose the Iterative Closest Point (ICP) algorithm to perform registration 
        between point sets, line segments, implicit curves, parametric curves, triangles, implicit surfaces, and parametric surfaces.
        ICP iteratively finds and updates correspondences between the desired sets based on an initial approximation to the solution and 
        the spatial distance.

        According to Segal et al. \cite{Segal_2009_generalizedicp}, the ICP algorithm can be summarized in two main steps:
        the computation of correspondences between the two scans and 
        the computation of a transformation that minimizes the distance between the corresponding points.
        Segal et al. present a generalized version of the ICP algorithm by attaching a probabilistic model to the second step.
        This generalization makes the algorithm more robust to outliers without modifying the simplicity. 
        However, it does not improve the speed of the original ICP.

        % ICP and its variants are still the standard algorithms for point cloud registration because of their simplicity. 
        % Nevertheless, to perform well they require a good initial approximation of the solution.

    %-------------------------------------------------------------------------------
    %	Hybrid methods
    %-------------------------------------------------------------------------------  
        \subsection{Hybrid methods} 

        The global methods results can be refined using a local method, most of the time using a variation of the popular ICP. 
        In this section, some of these methods that combine a global method with a local one can be found.

        Go-ICP is a branch-and-bound (BnB) based approach proposed by Yang et al. \cite{Yang_2016_goicp} to address the susceptibility of ICP to local minima. 
        The main idea of Go-ICP is that BnB and ICP work iteratively until a global minimum is reached: BnB finds a solution, which is then refined by ICP.
        According to the authors, Go-ICP guarantees global optimality regardless of the initialization, and it is recommended for scenarios where real-time performance is not critical.

        The approach proposed by Jun Lu et al. \cite{Lu_2019_4pcsicp} extracts keypoints from the point clouds, 
        then it obtains four optimal point correspondences from the key points by using Super4PCS \cite{Mellado_2014_super4pcs},
        i.e., the four-point correspondences that minimize the number of outliers. 
        Afterward, it creates a neighborhood ball with each point as its center and obtains these neighborhood balls’ overlapping regions. 
        Finally, it applies ICP to the overlapping regions. 
        The main drawback of this method is that overlapping regions are needed to perform the registration task.

        Due to the increasing success of Deep Learning in 2D applications, there have been attempts to use deep neural networks
        to solve the whole point cloud registration problem (global and local registration).
        DeepICP \cite{Lu_2019_deepicp} is, according to its creators, the first end-to-end learning-based point cloud registration framework.
        DeepICP makes use of other networks to build a complete framework.
        The first step of DeepICP is the Deep Feature Extraction (PointNet++ \cite{Qi_2017_pointnet}), which extracts features from the point clouds.
        In the second step, the Pont Weighting (3DFeatNet \cite{Yew_2018_3dfeat}) selects the relevant points.
        In the third step, the selected points of the target point cloud are sampled using the initial transformations.
        In the fourth step, Deep Feature Embedding is used to obtain more specific descriptors of the points.
        In the final step, the corresponding pairs of points are generated with a 3D Convolutional Neural Network
        to compute an improvement of the initial transformation.
        
        Other approaches do not use a neural network to solve the whole point cloud registration problem but part of it.
        LORAX is an algorithm proposed in \cite{Elbaz_2017_3dpoint} that selects some sets of points called super-points and uses a low-dimensional descriptor 
        generated by an autoencoder to describe the structure of each super-point.
        The sets of super-points from both point clouds are then matched using the euclidean distance between the descriptors.
        Finally, a coarse registration is computed using a RANSAC procedure, followed by a fine registration using ICP.
        The main contribution of LORAX is the use of an autoencoder to extract features from 3D point clouds. 
        Nevertheless, the implementation is not optimized for real-time performance, and it is not proved to be fast for large point clouds.

%-------------------------------------------------------------------------------
%	Point-to-model registration
%-------------------------------------------------------------------------------
    \section{Point-to-model registration}

    %-------------------------------------------------------------------------------
    %	Sampling methods
    %-------------------------------------------------------------------------------  
        \subsection{Sampling methods} 

        The most straightforward idea to register a 3D model with a point cloud is to sample points from the 3D model 
        to generate a second point cloud and then perform one of the point-to-point registration methods. 

        Kim et al. \cite{Kim_2011_fully} register a 3D CAD model of a construction site with a point cloud by sampling points from the 3D model 
        and applying a point-to-point registration algorithm. The registration consists of three main steps. First, 
        the point clouds are re-sampled to the same resolution. Second, PCA is used to identify the principal axes of the data 
        to obtain a coarse registration based on these axes. Third, fine registration is performed with the Levenberg-Marquardt
        iterative closest point (LM-ICP) algorithm \cite{Fitzgibbon_2003_robust}. 
        
        Kim et al. \cite{Kim_2013_fully} is an extension of their previous work \cite{Kim_2011_fully}.
        In this approach, they just add a noise filter step to the pre-processing of the point cloud to improve the registration results.
        The rest of the steps remain the same.
        However, using PCA for coarse registration will not assure a good performance when the building’s form resembles a cube.
        
    %-------------------------------------------------------------------------------
    %	Local methods
    %-------------------------------------------------------------------------------  
        \subsection{Local methods} 

        Other approaches work directly with the information provided in the 3D models to perform the registration task.

        Li and Song \cite{Li_2015_amodified} propose two extensions to the most popular local registration method, the ICP algorithm, 
        to use it with an STL-file (3D model) and a point cloud.
        The first of these extensions is called ICP-STL, and its main modification consists of finding the corresponding triangle mesh of the STL file to the points 
        in the point cloud, and then the closest point in the triangle mesh to each point of the point cloud. The rest of the ICP algorithm remains the same.
        The second modification is called ICP-DAF (ICP dynamic adjustment factor), and it aims to reduce the time of the ICP-STL method. 
        ICP-DAF adds a dynamic adjustment factor to adjust the transformation parameters dynamically.
        The disadvantage of this method is the same disadvantage present in the ICP algorithm, 
        i.e., to perform well, it requires a good initial approximation of the solution \cite{Sakakubara_2007_automatic}.

        In the specific case of CityGML, Goebbels et al. \cite{Goebbels_2019_icpcitygml} propose an extension of the well-known ICP
        to perform point-to-model registration based on point-to-plane registration.
        This approach only considers points on the perimeter of the buildings.
        The main contribution of this approach is the speed since point-to-model ICP is faster than the point-to-plane ICP.
        The CityGML models used, however, did not contain any terrain information.
        Therefore, according to the authors, the ground plane is the lowest point of the building, and some walls might intersect with the real terrain, increasing the algorithm’s error.
        An additional disadvantage of this method is that it expects the point cloud to be coarsely registered with the model.

        Another approach by Goebbels et al. \cite{Goebbels_2018_alinear} explores the possibility of finding features that help to perform the registration.
        The approach first rotates the point cloud to align the walls to the z-axis. Then, it projects all the points onto the 
        xy-plane to obtain a 2D image of the point cloud viewed from above. After that, it detects the corners in the 2D image.
        Having the CityGML model’s corners, it applies a Mixed Integer Linear Program to perform the registration.
        Finally, a Linear Program relaxation is used to fine-tune the coarse result. 
        Nonetheless, the point cloud has to be already coarsely registered to obtain a successful and fast result.

        An extension of the work in \cite{Goebbels_2018_alinear} has been made by Goebbels et al. \cite{Goebbels_2018_linebased}, 
        where the steps to perform the registration are the same, but instead of corners, lines are used in the Mixed Integer Linear Program. 
        Furthermore, no fine registration is performed at the end.
        Unfortunately, it has the same disadvantage as the previous approach: the point cloud has to be already coarsely registered to obtain a successful and fast result.
        However, this is not a disadvantage if the desired method will be just used to perform a fine registration.

%-------------------------------------------------------------------------------
%	Others
%-------------------------------------------------------------------------------      
    \section{Others} 

    Other approaches attempt to solve different matching problems with or without correspondences.
    These approaches cannot be classified in the previous categories because their input data is not composed of two point clouds or one point cloud and a 3D model.
    Some of these approaches are collected into this category.

    Breuel \cite{Breuel_2003_implementation} proposes the use of matchlist-based BnB techniques to solve geometric matching problems without correspondences.
    For example, point and line segment matching in 2D and 3D.

    Bazin et al. \cite{Bazin_2013_abranchandbound} present an approach that combines Linear Programming and BnB procedures to achieve global optimality.
    The proposed algorithm receives two different sets of features extracted from two images and outputs
    the correspondences between the features, verifying the appearance similarity and geometric constraints.
    In the end, it not only computes the feature correspondences but also computes an estimation of a geometric transformation between the features
    and identifies the inliers/outliers.

    Brown et al. \cite{Brown_2015_globally} extend the idea of the two previous approaches and present a 2D-3D registration method that does not need
    correspondences by using a BnB algorithm. They also propose a deterministic annealing algorithm to reduce computation time.
    The proposed method works with points or lines, but not with a combination of both.
    Therefore, Brown et al. \cite{Brown_2019_afamily} propose a framework to solve the 2D-3D registration problem without feature correspondences
    that is able to use points, lines, or a combination of both.

    Windheuser et al. \cite{Windheuser_2011_largescale} propose the use of Integer Linear Programming to find correspondences between non-rigid 3D shapes.
    They use an Integer Linear Program to minimize the elastic thin-shell energy required to deform one shape into the other one.
    Their method can be improved by a relaxation of the Integer Linear Program and the inclusion of feature descriptors.

%-------------------------------------------------------------------------------
%	Limitations of previous work
%-------------------------------------------------------------------------------
    \section{Limitations of previous work} 

    ICP and its variants provide simple and easily-implemented iterative methods, but these algorithms can converge to false local optima \cite{Wang_2019_deepclosest}.
    Furthermore, they do not scale well with the number of points, and they are not differentiable. 
    Therefore they cannot be integrated into end-to-end deep learning pipelines \cite{Sarode_2019_oneframework}. 
    Moreover, the principal disadvantage of these methods is that they require a good initial approximation to the solution to perform well \cite{Sakakubara_2007_automatic}.
    
    RANSAC and its variations cannot guarantee any global optimality in their final results \cite{Bazin_2013_abranchandbound}.
    According to \cite{Quan_2020_com}, RANSAC has two main issues. 
    The first one is its computational complexity, which makes it time-consuming.
    The second one is its randomness during sampling, which can lead to inaccurate results.
    These two disadvantages worsen with the number of outliers.

    The globally optimal methods, such as Linear programming and BnB based methods, are very time-consuming. 
    Therefore, their application in real-time tasks is limited \cite{Sarode_2019_oneframework}.

    In the last years, there has been a boom in the use of neural networks and deep neural networks to solve 2D problems.
    Therefore, their use in 3D tasks has been explored too.
    Unfortunately, methods based on deep learning generally suffer from issues regarding the generalization ability and the requirement of massive amounts of training data \cite{Quan_2020_com}.

    In general, none of these methods offer a straightforward solution for the coarse registration of a model with a point cloud.
    The only solution available for this is to sample a point cloud from the model and apply a global method for two point clouds.
    Unfortunately, this solution does not leverage the complete information contained in the model.
    However, some of the ideas of the approaches presented in this chapter can be used to implement alternative coarse registration methods of a point cloud with a CityGML model.

\end{document}
