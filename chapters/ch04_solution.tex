%!TEX root = ../report.tex

\begin{document}
    \chapter{Solution}
    Due to causes unrelated to this project, the available data was just one CityGML model and its corresponding point cloud.
    Therefore, a solution was designed based on this data and its characteristics.

    Later, in the middle phase of the project, another point cloud was available but its corresponding CityGML model was not.
    However, instead of a CityGML model, a 3D model of the site was available in Polygon File Format (ply).
    Because of the different information contained in the PLY file a different preprocessing was needed.

    Although (deep) learning methods are very attractive due to their boom in recent years,
    they are unthinkable to solve our problem. The main reason for this is the data needed to train such an approach.

    Goebbels et al. \cite{Goebbels_2018_linebased, Goebbels_2018_alinear} successfully propose the use of Mixed Integer Linear Programming to register
    a point cloud with a CityGML model, based on lines and points.
    Mixed Interger Linear Programming has been used to solve other registration problems
    like the registration between two point clouds \cite{Sakakubara_2007_automatic},
    registration between 3D shapes \cite{Windheuser_2011_largescale},
    and 2D registration \cite{Bazin_2013_abranchandbound}.

    Therefore, the solution proposed is to detect the walls in the point cloud and the 3D model by projecting both onto the xy-plane,  
    and then use the angles of the corners of the walls (where the walls intersect)
    in an Integer Linear Program to find correspondences and a transformation that aligns the point cloud with the 3D Model at the same time.
    The exact details are given in the next section.

    From now on, the point cloud could be referred to as the source, and the 3D model as the target.
     

    \section{Proposed algorithm}
    The main steps of the proposed algorithm are listed as follows:
    \begin{enumerate}
        \item Projection of the point cloud onto the xy-plane.
        \item Projection of the 3D model onto the xy-plane.
        \item Detection of lines.
        \item Detection of line intersections and their angles.
        \item Identification of possible correspondences between angles detected.
        \item Alignment of both projections onto the xy-plane.
        \item Translation across the z-axis.
    \end{enumerate}

    \subsection{Projection of the point cloud onto the xy-plane}
    \label{sub:Projection of the point cloud onto the xy-plane}
        To simplify the registration task, this is transported from the 3D space to the 2D space.
        The z-axis of the source and the target is already correctly aligned.
        This allows for a direct projection of the source onto the xy-plane to detect walls.

    \subsection{Projection the 3D model onto the xy-plane}
        The difference between the use of CityGML Models and the PLY models relies on their projection onto the xy-plane.
        From the PLY model, a source is sampled and projected onto the xy-plane as described in \autoref{sub:Projection of the point cloud onto the xy-plane}.
        
        The CityGML model already provides a terrain intersection.
        From this terrain intersection, only the x- and y-coordinates are used to make a projection onto the xy-plane.
        Lines are drawn from point to point in such a way that a footprint is obtained.

    \subsection{Detection of lines}
        The projections of the source and the target are now 2D images. 
        The source image could be noisy, but the target is clear and the lines that represent the walls are very well defined.
        Therefore, to deal with the noise and remark the lines of the source image an implementation of the technique proposed 
        by Chaudhuri et al. \cite{Chaudhuri_1989_detection}, to detect blood vessels in retinal images, together with a binary threshold is used. 
        The implementation of \cite{Chaudhuri_1989_detection} used is the one provided by DIPlib \cite{DIPlib_online}.

        Lines in both images are detected using the probabilistic hough line transform of OpenCV \cite{opencv_library}.
        The number of lines are reduced in three steps. 
        Firstly, lines with similar angle of inclination are detected.
        Secondly, from the lines that have a similar start and end position the longest one is preserved and the rest are deleted.
        Thirdly, lines that are inside other lines are deleted too.

    \subsection{Detection of line intersections and their angles}
        The lines are converted to the projective space and then cross product is used to detect the intersections between them.
        Only the intersections that are in the segments are taken into account.
        
        The angle of the intersections is computed.

    \subsection{Identification of possible correspondences between angles detected}
        The angles of the intersections in the source image and the target image are compared,
        and if they are similar to each other they are saved.

    \subsection{Alignment of both projections onto the xy-plane}
        

    \subsection{Translation across the z-axis}
        From the correspondences found, the point in the terrain intersection is found and the lowest point in the point cloud.

    \section{Implementation details}
    - Math

\end{document}
