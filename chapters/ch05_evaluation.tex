%!TEX root = ../report.tex

\begin{document}
    \chapter{Results and Evaluation}

    In this chapter, the results of the registration of the available data using the solution presented in \autoref{chap:Solution} will be presented and discussed.
    The data was visulized with the 3D visualization provided by Open3D.    

%-------------------------------------------------------------------------------
%	Registration of a CityGML model with a point cloud
%-------------------------------------------------------------------------------
    \section{Registration of a CityGML model with a point cloud}
        The initial pose of the CityGML model and the point cloud can be found in \autoref{fig:initial_CityGML}.
        \autoref{fig:initial_front_model} shows a front view of the fire exercise building in Dortmund, while \autoref{fig:initial_back_model} shows a back view.
        The point cloud is just rotated around the z-axis and translated across the x- and y-axis.
        It is also slightly translated across the z-axis, but this is not easy to visualize.

        After the registration process, the resulting transformation is

        \begin{equation*}
            T_1 = 
            \begin{bmatrix}0.987 & -0.197 & 0.0 & -4.994 \\ 
                0.197 & 0.987 & 0.0 & -2.636 \\
                0.0 & 0.0 & 1.0 & -0.207 \\
                0.0 & 0.0 & 0.0 & 1.0
            \end{bmatrix}
        \end{equation*}

        and the results after applying this transformation to the point cloud are showed in \autoref{fig:final_CityGML}.
        In the transformation, one can observe a rotation around the z-axis and a translation across the x-, y-, and z-axis.

        The alignment of the CityGML model with the point cloud can be seen in the fire exercise building (the one in the center of the image)
        and the right building in \autoref{fig:final_front_model}. The walls and the ceilings are correctly overlapped in both buildings.
        There is only a slight misalignment in the center part of the ceil of the fire exercise building because the CityGML model is outdated.

        \begin{figure}[H]
            \centering
            \begin{subfigure}{1\textwidth}
                \centering
                \includegraphics[scale=0.2]{images/solution_images/final_front.png}
                \caption{Front view.}
                \label{fig:final_front_model}
            \end{subfigure}
            \hfill
            \begin{subfigure}{1\textwidth}
                \centering
                \includegraphics[scale=0.2]{images/solution_images/final_back.png}
                \caption{Back view.}
                \label{fig:final_back_model}
            \end{subfigure}
            \caption{CityGML model and its corresponding point cloud after registration.}
            \label{fig:final_CityGML}
        \end{figure}

%-------------------------------------------------------------------------------
%	Registration of a PLY model with a point cloud
%-------------------------------------------------------------------------------
    \section{Registration of a PLY model with a point cloud}

        The second available data represents a part of the Technical University of Darmstadt.
        The 3D model is given in ply format, but this does not make any difference in the visualization.
        \autoref{fig:initial_ply} shows two different views of the initial pose of the 3D model and its corresponding point cloud.
        This time, the point cloud is translated across the x-, y-, and z-axis and rotated around the z-axis. 
        
        After the registration process, the resulting transformation is

        \begin{equation*}
            T_2 = 
            \begin{bmatrix}
                0.882 & -0.455 & 0.0 & 20.504 \\ 
                0.455 & 0.882 & 0.0 & -21.645 \\
                0.0 & 0.0 & 1.0 & -19.254 \\
                0.0 & 0.0 & 0.0 & 1.0
            \end{bmatrix}    
        \end{equation*}
        
        and the results after applying this transformation to the point cloud are showed in \autoref{fig:final_ply}.
        

        \begin{figure}[H]
            \centering
            \begin{subfigure}{1\textwidth}
                \centering
                \includegraphics[scale=0.2]{images/solution_images/initial_ply_a.png}
                \caption{Frist view.}
                \label{fig:intial_ply_a}
            \end{subfigure}
            \hfill
            \begin{subfigure}{1\textwidth}
                \centering
                \includegraphics[scale=0.2]{images/solution_images/initial_ply_b.png}
                \caption{Second view.}
                \label{fig:intial_ply_b}
            \end{subfigure}
            \caption{PLY model and its corresponding point cloud before registration.}
            \label{fig:initial_ply}
        \end{figure}

        In the transformation, one can observe a rotation around the z-axis and a translation across the x-, y-, and z-axis.
        This time, the translation across the z-axis is considerably larger than the one in $T_1$.

        In \autoref{fig:final_ply}, one can see that the walls are correctly aligned.
        Unfortunately, the point cloud does not contain any scans of the ceilings, and it is not clear whether the alignment in the z-axis is correct or not. 
        However, the point cloud contains the ground information of the center area between the buildings, 
        and one can see that it is aligned with the floor of the buildings.

        \begin{figure}[H]
            \centering
            \begin{subfigure}{1\textwidth}
                \centering
                \includegraphics[scale=0.2]{images/solution_images/final_ply_a.png}
                \caption{First view.}
                \label{fig:final_ply_a}
            \end{subfigure}
            \hfill
            \begin{subfigure}{1\textwidth}
                \centering
                \includegraphics[scale=0.2]{images/solution_images/final_ply_b.png}
                \caption{Second view.}
                \label{fig:final_ply_b}
            \end{subfigure}
            \caption{PLY model and its corresponding point cloud after registration.}
            \label{fig:final_ply}
        \end{figure}

%-------------------------------------------------------------------------------
%	Discussion
%-------------------------------------------------------------------------------
    \section{Discussion}
        The results presented in the two past sections depend on the step described in 
        \autoref{sub:Detection of line intersections and their angles}, i.e., on detecting lines and their intersections.
        If the number of intersections detected is poor, then the registration results could be wrong.
        And the accuracy of the registration result depends on the accuracy with which the position of the intersections is detected.
        
        The results obtained from the proposed solution are successful. 
        Although they just provide a coarse registration, this is significant for the proposes of the A-DRZ project.
        The registration method is ready to be included in the A-DRZ software of the Fraunhofer IAIS,
        and whenever there are new point clouds, the method could be further tested.
    
\end{document}
