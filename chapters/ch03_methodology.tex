%!TEX root = ../report.tex

\begin{document}
    \chapter{Methodology}

    In this section, a brief description of the Data and the definition of the Registration Problem will be given.
    CityGML Model structure is complex, and their complete explanation is out of the context of this project,
    For detailed information about CityGML please refer to Gröger et. al \cite{Groger_2012_OGC} or the official websites.

    \section{Data}
    The CityGML Models are public data in some states in Germany, for example in Nordrhein-Westfalen \cite{NRW3DModel_online}. 
    Other states of Germany like Hessen \cite{Hessen3DModel_online} or Bayern \cite{Bayern3DModel_online} charge money for their Models.
    That is not relevant for this project because the CityGML Models are received from the server of the A-DRZ project.

    The point cloud data is collected and provided by the A-DRZ project partners that develop and operate the actual Robots.
    These are mainly the University of Bonn \cite{UniBonn_online} and the Technical University of Darmstadt \cite{TUDarmstadt_online},
    which have integrated corresponding laser scanners and Lidar sensors on their platforms.    

    Unfortunately, there is no ground truth for the data available. Therefore, the results are evaluated manually by a person who decides whether the registration was successful or not.

    \section{Setup}
        - The input to the developed system is a CityGML model together with its corresponding point cloud.
        - The output is the visualization of the CityGML model aligned with its corresponding point cloud.
        - Define the problem using math
        -
    \section{Experimental Design}

    \section{Registration Problem}
        
\end{document}
